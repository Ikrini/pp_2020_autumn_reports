\documentclass{report}
\usepackage{tikz}
\usepackage[T2A]{fontenc}
\usepackage[utf8]{luainputenc}
\usepackage[english, russian]{babel}
\usepackage[pdftex]{hyperref}
\usepackage[14pt]{extsizes}
\usepackage{listings}
\usepackage{color}
\usepackage{geometry}
\usepackage{enumitem}
\usepackage{multirow}
\usepackage{graphicx}
\usepackage{indentfirst}

\geometry{a4paper,top=2cm,bottom=3cm,left=2cm,right=1.5cm}
\setlength{\parskip}{0.5cm}
\setlist{nolistsep, itemsep=0.3cm,parsep=0pt}

\lstset{language=C++,
		basicstyle=\footnotesize,
		keywordstyle=\color{blue}\ttfamily,
		stringstyle=\color{red}\ttfamily,
		commentstyle=\color{green}\ttfamily,
		morecomment=[l][\color{magenta}]{\#}, 
		tabsize=4,
		breaklines=true,
  		breakatwhitespace=true,
  		title=\lstname,       
}

\makeatletter
\renewcommand\@biblabel[1]{#1.\hfil}
\makeatother

\begin{document}

\begin{titlepage}

\begin{center}
Министерство науки и высшего образования Российской Федерации
\end{center}

\begin{center}
Федеральное государственное автономное образовательное учреждение высшего образования \\
Национальный исследовательский Нижегородский государственный университет им. Н.И. Лобачевского
\end{center}

\begin{center}
Институт информационных технологий, математики и механики
\end{center}

\vspace{4em}

\begin{center}
\textbf{\LargeОтчет по лабораторной работе} \\
\end{center}
\begin{center}
\textbf{\Large«Умножение плотных матриц. Элементы типа double. Блочная схема, алгоритм Фокса.»} \\
\end{center}

\vspace{4em}

\newbox{\lbox}
\savebox{\lbox}{\hbox{text}}
\newlength{\maxl}
\setlength{\maxl}{\wd\lbox}
\hfill\parbox{7cm}{
\hspace*{5cm}\hspace*{-5cm}\textbf{Выполнил:} \\ студент группы 381806-4 \\ Икромов И. И.\\
\\
\hspace*{5cm}\hspace*{-5cm}\textbf{Проверил:}\\ доцент кафедры МОСТ, \\ кандидат технических наук \\ Сысоев А. В.\\
}
\vspace{\fill}

\begin{center} Нижний Новгород \\ 2020 \end{center}

\end{titlepage}

\setcounter{page}{2}

% Содержание
\tableofcontents
\newpage



% Введение
\section*{Введение}
\addcontentsline{toc}{section}{Введение}
Матричное умножение является одной из существенных проблем в матричных
вычислениях. Оно определяется соотношением
 \begin{equation}\label{key}
	C_{ij} = \sum_{k=0}^{n-1} a_{i,k} \times\ b_{k,j} , \quad 0 \leq i,j < n
\end{equation}
\par Нетрудно заметить, что умножение матриц требует выполнения большого количества
операций - \begin{math} n^{3} \end{math} скалярных умножений и сложений. Что в свою очередь сильно сказывается на
эффективности программ, использующих данную операцию. Встает необходимость
распараллеливания вычислений.
\par При построении параллельных способов выполнения матричного умножения наряду с
рассмотрением матриц в виде наборов строк и столбцов широко используется блочное
представление матриц. При таком подходе матрицы-операнды A, B и результирующая
матрица C рассматриваются как набор блоков.
\par Разбиение матриц на блоки значительно упрощает проблему выбора эффективных
способов распараллеливания вычислений.
\par Одним из основных методов параллельного умножения матриц, использующий блочное
распределение данных между процессами, является алгоритм Фокса. Он и будет подробно
рассмотрен в настоящей работе.
\par Цель данной работы – реализовать параллельный алгоритм матричного умножения -
алгоритм Фокса, и сравнить его с последовательным алгоритмом


% Постановка задачи
\section*{Постановка задачи}
\addcontentsline{toc}{section}{Постановка задачи}
Даны две квадратные матрицы A и B размера n × n. Элементы матриц – вещественные
числа типа . Результатом умножения матриц A и B является матрица C размера n × n,
каждый элемент которой определяется в соответствии с выражением (1)
\par Требуется перемножить матрицы A и B задействовав P вычислительных узлов. В
качестве параллельного способа выполнения матричного умножения необходимо
использовать алгоритм Фокса при блочном разделении данных.

\par Считается, что число вычислительных улов является квадратом некоторого
натурального числа p и размер матриц кратен этому числу:
\begin{equation} P = p^{2}, \quad p \in N \quad \end{equation}
\begin{equation} \frac{n}{p} = z, \quad z \in N  \end{equation}
 \par Для реализации параллельной версии необходимо использовать средства MPI. Для проверки корректности работы алгоритмов требуется использовать Google C++ Testing Framework.
\newpage

% Метод решения
\section*{Метод решения}
\addcontentsline{toc}{section}{Метод решения}
В алгоритме Фокса используется блочная схема разбиения матриц - исходные матрицы
А, В и результирующая матрица С представляются в виде наборов блоков. Количество блоков
по горизонтали и вертикали одинаково и равно q (т.е. размер всех блоков равен k × k, k = n/q). 
\par Выполнение алгоритма Фокса включает: 
\\1. Этап инициализации.
\\2. Этап вычислений.
\\ На каждой итерации \[I: 0 \leq I < q\] осуществляются следующие операции:
\par \\2.1.1. Для каждой строки \[i: 0 \leq i < q\] блок \begin{math} A_{i,j} \end{math} подзадачи (i,j) пересылается на все подзадачи той же строки i решетки. Индекс j,определяющий положение подзадачи в строке, вычисляется в соответствии с выражением
\begin{equation} j = (i + I) mod \, q \end{equation}
Полученные в результаты пересылок блоки \begin{math}A'_{i,j}, B'_{i,j} \end{math} каждой подзадачи (i,j) перемножаются и прибавляются к блоку C_{i,j}.\end{math}
\par \\2.2.2. Блоки \begin{math}B'_{i,j} \end{math}каждой подзадачи (i,j) пересылаются подзадачам, являющимся соседями сверху в столбцах решетки подзадач (блоки подзадач из первой строки решетки пересылаются подзадачам последней строки решетки).
\newpage

% Схема распараллеливания
\section*{Схема распараллеливания}
\addcontentsline{toc}{section}{Схема распараллеливания}
ДРассмотрим основные моменты в организации параллельных вычислений для выполнения
алгоритма Фокса. 
\par 1. Построение топологии вычислительной системы.
\par \\ Для эффективного выполнения алгоритма Фокса, в котором базовые подзадачипредставлены в виде квадратной решетки размера q × q, множество имеющихся процессов 𝑃 также представляется в виде квадратной решетки размера p × p, это возможно.
\par \\ Так как число процессов в данной работе является полным квадратом, можно выбрать количество блоков в матрицах по вертикали и горизонтали равным p (т.е. q = p). Такой способ определения количества блоков приводит к тому, что объем вычислений в каждой подзадаче является одинаковым и тем самым достигается полная балансировка вычислительной нагрузки между процессами.
\par \\ Используя топологию вычислительной системы в виде квадратной решетки
размера p × p, производим отображение набора подзадач на множество процессов:
базовая подзадача (i,j) располагается на процессе P_{i,j}.

\par 2. Распределение памяти на процессах.
\par \\ В соответствии с алгоритмом Фокса, вычисления организованы так, что в
каждый текущий момент времени каждая подзадача (i,j) содержат лишь часть
необходимых для проведения расчетов данных, а доступ к остальной части данных
обеспечивается при помощи передачи данных между процессами. Таким образом, в
ходе вычислений на каждой базовой подзадаче (i,j) располагается четыре матричных
блока:
\par \\ - \quad  Блок \begin{math}C_{a,j} \end{math} матрицы C, вычисляемый подзадачей.
\par \\ - \quad  Блок \begin{math}A_{i,j} \end{math} матрицы A, размещаемый в подзадаче перед началом вычислений
\par \\ - \quad Блоки  \begin{math}A'_{i.j}, \, B'{i,j} \end{math} матриц A и B, получаемые подзадачей в ходе выполнения вычислений.
 
\newpage

% Описание программной реализации
\section*{Описание программной реализации}
\addcontentsline{toc}{section}{Описание программной реализации}
Рассмотрим подробно основные моменты программной реализации параллельного
алгоритма Фокса.
\par Описание глобальных переменных:
\par \quad  -> int procNum - число задействованных вычислительных узлов;
\par \quad -> int cartSize - размер решетки процессов;
\par \quad -> int blockSize - размер матричных блоков;
\par \quad -> int rank - ранк процесса;
\par \quad -> int coords[2] - координаты процесса в решетке;
\par \quad -> MPI_Comm _ comm _ cart - grid switch;
\par \quad -> MPI_Comm _ comm _ row - string communicators;
\par \quad -> MPI_Comm_comm_col - column communicators;

\par Описание основных методов:

\begin{lstlisting}
void Topology()
\end{lstlisting}
создает коммуникатор в виде двумерной квадратной решетки
(используя функцию MPI_Cart_create), определяет координаты каждого процесса в этой решетке (MPI_Cart_coords), а также создает коммуникаторы отдельно для каждой строки и каждого столбца решетки(MPI_Cart_sub).
\begin{lstlisting}
void Scatter(double* Matrix, double* block, int size)
\end{lstlisting}
обеспечивает выполнение блочного распределения матрицы Matrix размера size × size между процессами, организованными в двумерную квадратную решетку. Данные рассылаются в два этапа. На первом этапе матрица Matrix, находящаяся в «корневом» процессе с координатами (0,0), разделяется на горизонтальные полосы. Эти полосы распределяются на процессы, составляющие нулевой столбец решетки процессов (MPI_Scatter для коммуникатора нулевого столбца). Далее каждая полоса разделяется на блоки между процессами, составляющими строки решетки процессов (MPI_Scatter для каждого коммуникатора строк). Полученные на процессах блоки записываются в буфер block. В программе этот метод реализует распределение матрицоперандов 𝐴 и 𝐵 между процессами.
\begin{lstlisting}
void ABlockTransfer(int it, double* Ablock, double* AMatrixblock) 
\end{lstlisting}
выполняет рассылку блоков матрицы A по строкам решетки процессов в буфер приема Ablock. Для этого в каждой строке решетки определяется ведущий процесс, осуществляющий рассылку. Для рассылки используется блок AMatrixblock, переданный в процесс в момент начального распределения данных. Выполнение операции рассылки блоков осуществляется при помощи функции MPI_Bcast. Следует отметить, что данная операция является коллективной и ее 8 локализация пределами отдельных строк решетки обеспечивается за счет использования
коммуникаторов comm_row, определенных для набора процессов каждой строки решетки в отдельности.
\begin{lstlisting}
void BlockMultiplication(double* Ablock, double* Bblock, double* Cblock, int _BlockSize)
\end{lstlisting}
обеспечивает перемножение блоков Ablock и Bblock, размерами _BlockSize в соответствии с формулой (1). Результат умножения записывается в Cblock.
\begin{lstlisting}
void BblockTransfer(double* Bblock) 
\end{lstlisting}
выполняет циклический сдвиг блоков матрицы 𝐵по столбцам процессорной решетки вверх. Каждый процесс передает свой блок следующемупроцессу, расположенному выше в столбце решетки процессов, и получает блок, переданныйиз процесса, расположенного ниже. Выполнение операций передачи данных осуществляетсяпри помощи функции MPI_SendRecv_replace, которая обеспечивает все необходимыепересылки блоков, используя при этом один и тот же буфер памяти Bblock.
\begin{lstlisting}
void FoxAlg(double* Ablock, double* AMatrixblock, double* Bblock, double* Cblock)
\end{lstlisting}
реализует логику работы параллельного алгоритма Фокса, описанную в главе «Методы решения»: В цикле от 0 до GridSize происходит рассылка блоков AMatrixblock, по строкам процессной решетки в буферы Ablock (функция ABlockTransfer), затем выполняется умножение блоков (BlockMultiplication). Результат умножения записывается в Cblock. После чего осуществляется циклический сдвиг блоков Bblock в столбцах процессной решетки (BblockTransfer).

\begin{lstlisting}
void ImportResult(double* C, double* Cblock, intsize)
\end{lstlisting}
обеспечивает сбор результирующей матрицы 𝐶 размера 𝑠𝑖𝑧𝑒 × 𝑠𝑖𝑧𝑒 из блоков Cblock. Сбор данных также выполняется при помощи двухэтапной процедуры, зеркально отображающей процедуру распределения матрицы Scatter.
\begin{lstlisting}
void Fox(double* A, double* B, double* C, int size)
\end{lstlisting}
главная функция программы. Определяет основную логику работы алгоритма, последовательно вызывает необходимые подпрограммы, описанные выше. В качестве аргументов принимает матрицы A и B участвующие в умножении, матрицу C для записи результата и их размер .
\newpage

% Подтверждение корректности
\section*{Подтверждение корректности}
\addcontentsline{toc}{section}{Подтверждение корректности}
Для подтверждения корректности в программе представлен набор тестов, разработанных с помощью использования Google C++ Testing Framework.
\par Набор представляет из себя тесты, которые проверяют корректность входных данных (вектор, который необходимо отсортировать, должен иметь положительный размер), корректность вычислений (сравнивается полученный благодаря параллельной или последовательной сортировке вектор с вектором, отсортированным с помощью функции сортировки из STL библиотеки), а также эффективность (вычисление занимаемого последовательной и параллельной сортировок времени и сравнение полученных данных).
\par Успешное прохождение всех тестов доказывает корректность работы программного комплекса.
\newpage

% Результаты экспериментов
\section*{Результаты экспериментов}
\addcontentsline{toc}{section}{Результаты экспериментов}
Эксперименты проводились на ПК со следующими параметрами:

\par В рамках эксперимента было вычислено время работы параллельного (алгоритм Фокса) и последовательно алгоритмов умножения квадратных матриц A и B. Размер матриц изменяется в диапазоне от [500; 2500] с шагом 500. Элементами матриц являются вещественные числа типа double. Эксперименты проводятся на 4, 9 и 49 процессах. Время выражено в секундах. Ускорение рассчитывается по формуле: \begin{equation} Ускорение = время последовательного алгоритма / время параллельного алгоритма \end{equation} 

\begin{itemize}
	\item Процессор: Intel Core 5-7400, 3000 MHz, ядер: 4;
	\item Оперативная память: 8192 МБ (DDR4), 2400 MHz;
	\item ОС: Microsoft Windows 10 Home, версия 1909 сборка 18363.1198.
\end{itemize}

\par Для проведения экспериментов производилась сортировка векторов размером \verb|20 000 000| значений каждый. 
\par Результаты экспериментов представлены в Таблице 1.

\begin{table}[!h]
	\caption{Результаты вычислительных экспериментов}
	\centering
	\begin{tabular}{lllll}
		Размер матриц & Последовательно & Параллельно & Ускорение \\
		500             & 0.637         & 0.236     & 2.699       \\
		1000            & 8.132         & 3.707     & 2.194       \\
		1500            & 23.189        & 12.065    & 2.0194      \\
		2000            & 62.404        & 32.366    & 1.928       \\
		2500            & 159.031       & 72.379    & 1.928       \\
	\end{tabular}
\end{table}

\par По данным экспериментов видно, что алгоритм Фокса работает намного быстрее, чем классический алгоритм умножения матриц. Наибольшая эффективность достигается при достаточно большом размере матриц - размер матриц должен превышать 1000 элементов.
\par Также заметим, что размер матриц и число процессов должны быть прямо пропорциональны друг другу, для поддержки наилучшей эффективности. Так, из графика ускорения видно, что при малом размере матриц 49 процессов дают низкую эффективность из-за больших латентностей, возникающих при передаче данных между процессами. В то время как операции умножения блоков выполняются быстро. 
\par Таким образом, можно сделать вывод: чем меньше матрица, тем меньшее число процессов нужно использовать. И наоборот, чем больше матрица, тем большее число процессов необходимо для эффективного умножения.
\newpage

% Заключение
\section*{Заключение}
\addcontentsline{toc}{section}{Заключение}
В результате выполнения лабораторной работы была разработана библиотека, реализующая параллельный метод матричного умножения – алгоритм Фокса, используя технологию MPI.
\par Для подтверждения корректности работы программы разработан и доведен до успешного выполнения набор тестов с использованием библиотеки модульного тестирования Google C++ Testing Framework.
\par По данным экспериментов удалось сравнить время работы параллельного и последовательного алгоритмов умножения матриц. Выявлено, что параллельный алгоритм Фокса показывает высокую эффективность на достаточно большом объеме данных. И число задействованных вычислительных узлов должно быть прямо пропорционально размеру матриц, участвующих в умножении. 
\newpage

% Список литературы
\begin{thebibliography}{1}
	\addcontentsline{toc}{section}{Список литературы}
	\bibitem{Sidnev} Сиднев А.А., Сысоев А.В., Мееров И.Б «Лабораторная работа №7. Оптимизация вычислительно трудоемкого программного модуля для архитектуры Intel Xeon Phi. Линейные сортировки». Нижний Новгород, 2007, 56 с. 
	\bibitem{parallel} Parallel: Лаборатория Параллельных информационных технологий НИВЦ МГУ [Электронный ресурс] // URL: \url {https://parallel.ru/vvv/mpi.html} (дата обращения: 28.11.2020)
	\bibitem{Algolist} Algolist: Алгоритмы, методы, исходники [Электронный ресурс] // URL: \url {http://algolist.manual.ru/sort/radix_sort.php} (дата обращения: 28.11.2020)
\end{thebibliography}
\newpage

% Приложение
\section*{Приложение}
\addcontentsline{toc}{section}{Приложение}
В данном разделе находится листинг всего кода, написанного в рамках лабораторной работы.
\begin{lstlisting}
// fox.h
// Copyright 2020 Ikromov Inom

#ifndef MODULES_TASK_3_IKROMOV_I_FOX_FOX_H_
#define MODULES_TASK_3_IKROMOV_I_FOX_FOX_H_

void Fox(double* A, double* B, double* C, int size);
void SequentialAlgorithm(double* A, double* B, double* C, int size);
void RandomOperandMatrix(double* A, double* B, int size);

#endif  // MODULES_TASK_3_IKROMOV_I_FOX_FOX_H_
\end{lstlisting}

\begin{lstlisting}
// fox.cpp
// Copyright 2020 Ikromov Inom

#include <mpi.h>
#include <cmath>
#include <ctime>
#include <random>
#include <stdexcept>
#include "../../../modules/task_3/ikromov_i_fox/fox.h"

#define ndims 2
#define reoder true

int procNum;
int rank;
int cartSize;
int blockSize;
MPI_Comm comm_cart;
MPI_Comm comm_row;
MPI_Comm comm_col;
int coords[2];

void ImportResult(double* C, double* Cblock, int size) {
double* rowbuff = new double[size * blockSize];
for (int i = 0; i < blockSize; i++) {
MPI_Gather(&Cblock[i * blockSize], blockSize, MPI_DOUBLE,
&rowbuff[i * size], blockSize, MPI_DOUBLE, 0, comm_row);
}
if (coords[1] == 0) {
MPI_Gather(rowbuff, blockSize * size, MPI_DOUBLE, C, blockSize * size,
MPI_DOUBLE, 0, comm_col);
}
delete[] rowbuff;
}

void Scatter(double* Matrix, double* block, int size) {
double* rowbuff = new double[blockSize * size];
if (coords[1] == 0) {
MPI_Scatter(Matrix, blockSize * size, MPI_DOUBLE, rowbuff, blockSize * size,
MPI_DOUBLE, 0, comm_col);
}
for (int i = 0; i < blockSize; i++) {
MPI_Scatter(&rowbuff[i * size], blockSize, MPI_DOUBLE,
&(block[i * blockSize]), blockSize, MPI_DOUBLE, 0, comm_row);
}
delete[] rowbuff;
}

void BlockExport(double* A, double* B, double* AMatrixblock, double* Bblock,
int size) {
Scatter(A, AMatrixblock, size);
Scatter(B, Bblock, size);
}

void ABlockTransfer(int it, double* Ablock, double* AMatrixblock) {
int main = (coords[0] + it) % cartSize;
if (coords[1] == main) {
for (int i = 0; i < blockSize * blockSize; ++i) Ablock[i] = AMatrixblock[i];
}
MPI_Bcast(Ablock, blockSize * blockSize, MPI_DOUBLE, main, comm_row);
}

void BlockMultiplication(double* Ablock, double* Bblock, double* Cblock,
int _blockSize) {
double temp;
for (int i = 0; i < _blockSize; i++)
for (int j = 0; j < _blockSize; j++) {
temp = 0;
for (int k = 0; k < _blockSize; k++)
temp += Ablock[i * _blockSize + k] * Bblock[k * _blockSize + j];
Cblock[i * _blockSize + j] += temp;
}
}

void BblockTransfer(double* Bblock) {
MPI_Status status;
int dest = coords[0] - 1;
if (coords[0] == 0) dest = cartSize - 1;
int source = coords[0] + 1;
if (coords[0] == cartSize - 1) source = 0;
MPI_Sendrecv_replace(Bblock, blockSize * blockSize, MPI_DOUBLE, dest, 0,
source, 0, comm_col, &status);
}

void FoxAlg(double* Ablock, double* AMatrixblock, double* Bblock,
double* Cblock) {
for (int it = 0; it < cartSize; ++it) {
ABlockTransfer(it, Ablock, AMatrixblock);
BlockMultiplication(Ablock, Bblock, Cblock, blockSize);
BblockTransfer(Bblock);
}
}

void Topology() {
int dimSize[2] = { cartSize, cartSize };
int periods[2] = { false, false };
MPI_Cart_create(MPI_COMM_WORLD, ndims, dimSize, periods, reoder, &comm_cart);
MPI_Cart_coords(comm_cart, rank, ndims, coords);

int subdims[2];
subdims[0] = false;
subdims[1] = true;
MPI_Cart_sub(comm_cart, subdims, &comm_row);

subdims[0] = true;
subdims[1] = false;
MPI_Cart_sub(comm_cart, subdims, &comm_col);
}

void Fox(double* A, double* B, double* C, int size) {
if (size <= 0) throw std::runtime_error("Wrong size matrix");

MPI_Comm_size(MPI_COMM_WORLD, &procNum);
MPI_Comm_rank(MPI_COMM_WORLD, &rank);

if (procNum == 1) {
SequentialAlgorithm(A, B, C, size);
} else {
cartSize = static_cast<int>(std::sqrt(procNum));
if (cartSize * cartSize != procNum)
throw std::runtime_error("Wrong number of processes");
if (size % cartSize != 0) throw std::runtime_error("Wrong size matrix");
blockSize = size / cartSize;

int numElemBlock = blockSize * blockSize;
double* Ablock = new double[numElemBlock];
double* Bblock = new double[numElemBlock];
double* Cblock = new double[numElemBlock];
double* AMatrixblock = new double[numElemBlock];
for (int i = 0; i < numElemBlock; i++) Cblock[i] = 0;

Topology();

BlockExport(A, B, AMatrixblock, Bblock, size);

FoxAlg(Ablock, AMatrixblock, Bblock, Cblock);

ImportResult(C, Cblock, size);

delete[] Ablock;
delete[] Bblock;
delete[] Cblock;
delete[] AMatrixblock;
}
}

void SequentialAlgorithm(double* A, double* B, double* C, int size) {
if (size <= 0) throw std::runtime_error("Wrong size matrix");
for (int i = 0; i < size * size; ++i) C[i] = 0;
BlockMultiplication(A, B, C, size);
}

void RandomOperandMatrix(double* A, double* B, int size) {
std::mt19937 gen;
gen.seed(static_cast<unsigned int>(time(0)));
for (int i = 0; i < size * size; ++i) {
A[i] = gen() % 5 + static_cast<float>(gen() % 10) / 10;
B[i] = gen() % 5 + static_cast<float>(gen() % 10) / 10;
}
}
\end{lstlisting}
\begin{lstlisting}
// main.cpp
// Copyright 2020 Ikromov Inom

#include <gtest-mpi-listener.hpp>
#include <gtest/gtest.h>
#include <cmath>
#include "../../../modules/task_3/ikromov_i_fox/fox.h"

TEST(Fox, Test_on_Matrix_size_4) {
int size = 4;
double tmp;
double* A = &tmp;
double* B = &tmp;
double* C = &tmp;
double* CFox = &tmp;
int rank, procNum;
MPI_Comm_rank(MPI_COMM_WORLD, &rank);
MPI_Comm_size(MPI_COMM_WORLD, &procNum);

int cartSize = static_cast<int>(std::sqrt(procNum));
if (cartSize * cartSize == procNum && size % cartSize == 0) {
if (rank == 0) {
A = new double[size * size];
B = new double[size * size];
RandomOperandMatrix(A, B, size);
C = new double[size * size];
CFox = new double[size * size];
SequentialAlgorithm(A, B, C, size);
}
Fox(A, B, CFox, size);

if (rank == 0) {
for (int i = 0; i < size * size; ++i) ASSERT_NEAR(CFox[i], C[i], 0.0001);
}

if (rank == 0) {
delete[] A;
delete[] B;
delete[] C;
delete[] CFox;
}
}
}

TEST(Fox, Test_on_Matrix_size_16) {
int size = 16;
double tmp;
double* A = &tmp;
double* B = &tmp;
double* C = &tmp;
double* CFox = &tmp;
int rank, procNum;
MPI_Comm_rank(MPI_COMM_WORLD, &rank);
MPI_Comm_size(MPI_COMM_WORLD, &procNum);

int cartSize = static_cast<int>(std::sqrt(procNum));
if (cartSize * cartSize == procNum && size % cartSize == 0) {
if (rank == 0) {
A = new double[size * size];
B = new double[size * size];
RandomOperandMatrix(A, B, size);
C = new double[size * size];
CFox = new double[size * size];
SequentialAlgorithm(A, B, C, size);
}
Fox(A, B, CFox, size);

if (rank == 0) {
for (int i = 0; i < size * size; ++i) ASSERT_NEAR(CFox[i], C[i], 0.0001);
}

if (rank == 0) {
delete[] A;
delete[] B;
delete[] C;
delete[] CFox;
}
}
}

TEST(Fox, Test_on_Matrix_size_64) {
int size = 64;
double temp;
double* A = &temp;
double* B = &temp;
double* C = &temp;
double* CFox = &temp;
int rank, procNum;
MPI_Comm_rank(MPI_COMM_WORLD, &rank);
MPI_Comm_size(MPI_COMM_WORLD, &procNum);

int cartSize = static_cast<int>(std::sqrt(procNum));
if (cartSize * cartSize == procNum && size % cartSize == 0) {
if (rank == 0) {
A = new double[size * size];
B = new double[size * size];
RandomOperandMatrix(A, B, size);
C = new double[size * size];
CFox = new double[size * size];
SequentialAlgorithm(A, B, C, size);
}
Fox(A, B, CFox, size);

if (rank == 0) {
for (int i = 0; i < size * size; ++i) ASSERT_NEAR(CFox[i], C[i], 0.0001);
}

if (rank == 0) {
delete[] A;
delete[] B;
delete[] C;
delete[] CFox;
}
}
}

TEST(Fox, Throw_Matrix_size_0) {
int size = 0;
double tmp;
double* M = &tmp;
ASSERT_ANY_THROW(Fox(M, M, M, size));
ASSERT_ANY_THROW(SequentialAlgorithm(M, M, M, size));
}

TEST(Fox, Throw_Matrix_size_11) {
int size = 13;
double tmp;
double* M = &tmp;
int procNum;
MPI_Comm_size(MPI_COMM_WORLD, &procNum);

if (procNum > 1) {
ASSERT_ANY_THROW(Fox(M, M, M, size));
}
}

int main(int argc, char** argv) {
::testing::InitGoogleTest(&argc, argv);
MPI_Init(&argc, &argv);

::testing::AddGlobalTestEnvironment(new GTestMPIListener::MPIEnvironment);
::testing::TestEventListeners& listeners =
::testing::UnitTest::GetInstance()->listeners();

listeners.Release(listeners.default_result_printer());
listeners.Release(listeners.default_xml_generator());

listeners.Append(new GTestMPIListener::MPIMinimalistPrinter);
return RUN_ALL_TESTS();
}
listeners.Append(new GTestMPIListener::MPIMinimalistPrinter);
return RUN_ALL_TESTS();
}
\end{lstlisting}

\end{document}

